%!TEX root = derivation.tex

\section{ Global Hidden Markov Model の変分ベイズ解法 }
\label{sec:globalVarBayesHmm}

$R$ 本の時系列データからグローバルに解析する方法。
パラメータは共通とする。

データ ${\bf X}$ が、
\begin{align}
  {\bf X}  =\;&  \{ {\bf X}^{(1)}, {\bf X}^{(2)}, \dots, {\bf X}^{(R)} \}  \\
  {\bf X}^{(r)}  =\;&  \{ x_1^{(r)}, x_2^{(r)}, \dots, x_{N^{(r)}}^{(r)} \}
\end{align}
とする。
ただし、$1 \le r \le R$。
潜在変数 ${\bf Z}$ も同様に
\begin{align}
  {\bf Z}  =\;&  \{ {\bf Z}^{(1)}, {\bf Z}^{(2)}, \dots, {\bf Z}^{(R)} \}  \\
  {\bf Z}^{(r)}  =\;&  \{ {\bf z}_1^{(r)}, {\bf z}_2^{(r)}, \dots, {\bf z}_{N^{(r)}}^{(r)} \}
\end{align}
に拡張する。

この時 ${\bf X}, {\bf Z}, \theta$ の同時分布は、
\begin{align}
  p( {\bf X}, {\bf Z}, \theta )  =\;&  p(\theta) \cdot p( {\bf X}, {\bf Z} | \theta )  \\
%
  =\;&  p(\theta) \cdot \prod_{r=1}^R  p( {\bf X}^{(r)}, {\bf Z}^{(r)} | \theta )  \\
%
  =\;&  p(\theta) \cdot \prod_{r=1}^R \left\{  p( {\bf z}_1^{(r)} | \theta ) \times \prod_{n=2}^{N^{(r)}} p( {\bf z}_n^{(r)} | {\bf z}_{n-1}^{(r)}, \theta ) \times \prod_{m=1}^{N^{(r)}} p( x_m^{(r)} | {\bf z}_m^{(r)}, \theta )  \right\}  
\end{align}

\begin{align}
  \ln p( {\bf X}, {\bf Z}, \theta )  =\;&  \ln p(\theta)  %
    +  \sum_{r=1}^R \Biggl\{  %
      \ln p( {\bf z}_1^{(r)} | \theta )  %
      + \sum_{n=2}^{N^{(r)}} \ln p( {\bf z}_n^{(r)} | {\bf z}_{n-1}^{(r)}, \theta )  %
      + \sum_{m=1}^{N^{(r)}} \ln p( x_m^{(r)} | {\bf z}_m^{(r)}, \theta )  %
    \Biggr\}  
\end{align}


\subsection{ E-step }

$\theta$ の平均場近似で $q^*({\bf Z})$ を計算すると、

\begin{align}
  \ln q^*( {\bf Z} )  =\;&  \mathbb{E}_{\theta}\left[  \sum_{r=1}^R \left\{ \ln p( {\bf z}_1^{(r)} | \theta )  %
      + \sum_{n=2}^{N^{(r)}} \ln p( {\bf z}_n^{(r)} | {\bf z}_{n-1}^{(r)}, \theta )  %
      + \sum_{m=1}^{N^{(r)}} \ln p( x_m^{(r)} | {\bf z}_m^{(r)}, \theta )  %
    \right\}  \right]  +  {\rm const.}  \\
%
  =\;&  \mathbb{E}_{\theta}\left[  \sum_{r=1}^R  \ln p( {\bf z}_1^{(r)} | \theta )  \right]  %
      + \mathbb{E}_{\theta}\left[  \sum_{r=1}^R  \sum_{n=2}^{N^{(r)}} \ln p( {\bf z}_n^{(r)} | {\bf z}_{n-1}^{(r)}, \theta )  \right]  \notag  \\  %
  &  + \mathbb{E}_{\theta}\left[  \sum_{r=1}^R  \sum_{m=1}^{N^{(r)}} \ln p( x_m^{(r)} | {\bf z}_m^{(r)}, \theta )  \right]  %
    +  {\rm const.}  
%
\intertext{$p( {\bf z}_1^{(r)} | \theta ) = \prod_{i=1}^K \pi_i^{z_{1i}^{(r)}}$, $p( {\bf z}_n^{(r)} | {\bf z}_{n-1}^{(r)}, \theta ) = \prod_{i=1}^K \prod_{j=1}^K A_{ij}^{z_{n-1,i}^{(r)},z_{nj}^{(r)}}$ とすると、}
%
  =\;&  \mathbb{E}_{\pi}\left[  \sum_{r=1}^R  \sum_{i=1}^K z_{1i}^{(r)} \ln \pi_i  \right]  %
      + \mathbb{E}_{A}\left[  \sum_{r=1}^R  \sum_{n=2}^{N^{(r)}} \sum_{i=1}^K \sum_{j=1}^K  z_{n-1,i}^{(r)} z_{nj}^{(r)} \ln A_{ij}  \right]  \notag  \\  %
  &  + \mathbb{E}_{\theta}\left[  \sum_{r=1}^R  \sum_{m=1}^{N^{(r)}} \ln p( x_m^{(r)} | {\bf z}_m^{(r)}, \theta )  \right]  %
    +  {\rm const.}  \\
%
  =\;&  \sum_{r=1}^R  \left\{  \sum_{i=1}^K z_{1i}^{(r)} \overline{\ln \pi_i}  %
      + \sum_{n=2}^{N^{(r)}} \sum_{i=1}^K \sum_{j=1}^K  z_{n-1,i}^{(r)} z_{nj}^{(r)} \overline{\ln A_{ij}}  %
      + \sum_{m=1}^{N^{(r)}} \overline{\ln p( x_m^{(r)} | {\bf z}_m^{(r)}, \theta )}  %
    \right\}+  {\rm const.}  \\
%
\intertext{したがって、}
%
  q^*({\bf Z})  \propto\;&  \prod_{r=1}^R  \times  \Biggl\{  %
    \prod_{i=1}^K  \exp\left(\overline{\ln \pi_i}\right)^{z_{1i}^{(r)}}  %
    \times  \prod_{n=2}^{N^{(r)}} \prod_{i=1}^K \prod_{j=1}^K  \exp\left(\overline{\ln A_{ij}}\right)^{z_{n-1,i}^{(r)} z_{nj}^{(r)}}  \notag  \\  %
  &  \times  \prod_{m=1}^{N^{(r)}} \exp\left(\overline{\ln p( x_m^{(r)} | {\bf z}_m^{(r)}, \theta )}\right)  %
  \Biggr\}  \\
\end{align}

各 $r$ について独立なので、単独の HMM と同様にそれぞれ $\gamma^{(r)}, \xi^{(r)}$ を求める。


\subsection{ M-step }

各パラメータについて分布関数 $q$ を更新し、期待値を求める。

${\bf X}, {\bf Z}, \theta$ の同時分布
\begin{align}
  p( {\bf X}, {\bf Z}, \theta )  =\;&  p(\theta) \cdot p( {\bf X}, {\bf Z} | \theta )  \\
%
  =\;&  p(\theta) \cdot \prod_{r=1}^R \left\{  p\left( {\bf X}^{(r)}, {\bf Z}^{(r)} | \theta \right)  \right\}  \\
%
  =\;&  p(\theta) \cdot \prod_{r=1}^R \left\{  p\left( {\bf z}_1^{(r)} | \theta \right) \times \prod_{n=2}^{N^{(r)}} p\left( {\bf z}_n^{(r)} | {\bf z}_{n-1}^{(r)}, \theta \right) \right. \notag  \\
  &  \left. \times \prod_{m=1}^{N^{(r)}} p\left( x_m^{(r)} | {\bf z}_m^{(r)}, \theta \right)  \right\}  \\
%
\intertext{ から、${\bf Z}$ を固定する平均場近似を用いて }
%
  \ln q^*( \theta )  &=  \mathbb{E}_{\bf Z} \left[  \ln p( \theta ) +  \sum_{r=1}^R  \left\{  %
    \ln p\left( {\bf z}_1^{(r)} \right)  \right.\right. \notag  \\
    &  \left.\left. +  \sum_{n=2}^N \ln p\left( {\bf z}_n^{(r)} | {\bf z}_{n-1}^{(r)} \right)  %
    +  \sum_{m=1}^N \ln p\left( x_m^{(r)} | {\bf z}_m^{(r)} \right)  %
    \right\}  \right]  
\end{align}
により $q^*$ を得る。

\

パラメータ ${\boldsymbol \pi}$, ${\bf A}$, $\phi$ がそれぞれ独立、かつ、$p( x_m^{(r)} | {\bf z}_m^{(r)}, \theta )$ は ${\boldsymbol \pi}$, ${\bf A}$ に依存しないとすると、まず ${\boldsymbol \pi}$ に関して、
\begin{align}
  \ln q^*( {\boldsymbol \pi} )  &=  \ln p( {\boldsymbol \pi} ) + \mathbb{E}_{\bf Z} \left[ \sum_{r=1}^R \ln p\left( {\bf z}_1^{(r)} \right) \right] + {\rm const.}  \\
%
  &=  \ln p( {\boldsymbol \pi} ) + \mathbb{E}_{\bf Z} \left[ \sum_{r=1}^R  \sum_i^K z_{1i}^{(r)} \ln \pi_i \right] + {\rm const.}  \\
%
  &=  \ln p( {\boldsymbol \pi} ) + \sum_i^K \left( \sum_r^R \overline{z_{1i}^{(r)}} \right) \ln \pi_i + {\rm const.}  
%
\intertext{ となり、 }
%
  q^*( {\boldsymbol \pi} )  &\propto  p( {\boldsymbol \pi} ) \times \prod_i^K {\pi_i}^{ \overline{z_{1i}^R} }  %\\
%
\intertext{ ただし、$\overline{z_{1i}}^R = \sum_r^R \overline{z_{1i}}^{(r)}$。
ここで、事前分布として Dirichlet 分布 $p( {\boldsymbol \pi} ) \propto \prod_i^K {\pi_i}^{(u_i^\pi - 1)}$ を与えて規格化して、 }
%
  q^*( {\boldsymbol \pi} )  &=  \frac{ \Gamma\left( u_0^\pi + R \right) }{\displaystyle  \prod_i^K \Gamma\left( u_i^\pi + \overline{z_{1i}^R} \right) }\prod_i^K {\pi_i}^{\left( u_i^\pi + \overline{z_{1i}^R} - 1 \right)}  %\label{eqn:qPi}  
\end{align}
ただし、$u_i^\pi$ は Dirichlet 分布のハイパーパラメータ。
特に理由が無ければ、$u_i^\pi = 1$ としておけばよい。

Dirichlet 分布の性質から、期待値
\begin{align}
  \overline{\pi_i}  &=  \frac{ u_i^\pi + \overline{z_{1i}^R} }{ u_0^\pi + R }  \\  %\label{eqn:avgPi}  \\
  \overline{\ln \pi_i}  &=  \psi \left( u_i^\pi + \overline{z_{1i}^R} \right) - \psi\left( u_0^\pi + R \right)  %\label{eqn:avgLnPi}  
\end{align}
を得る。
%ただし、$u_0^\pi = \sum_i^K u_i^\pi$、$\psi(x) = \frac{d}{dx} \ln \Gamma(x) = \frac{ \Gamma'(x) }{ \Gamma(x) }$ は digamma 関数。


次に ${\bf A}$ に関して、
\begin{align}
  \ln q^*( {\bf A} )  &=  \ln p( {\bf A} ) + \mathbb{E}_{\bf Z} \left[ \sum_{r=1}^R  \sum_{n=2}^{N^{(r)}} \ln p( {\bf z}_n^{(r)} | {\bf z}_{n-1}^{(r)} ) \right] + {\rm const.}  \\
%
  &=  \ln p( {\bf A} ) + \mathbb{E}_{\bf Z} \left[ \sum_{r=1}^R  \sum_{n=2}^{N^{(r)}}  \sum_i^K \sum_j^K z_{n-1, i}^{(r)} z_{nj}^{(r)} \ln A_{ij} \right] + {\rm const.}  \\
%
  &=  \ln p( {\bf A} ) + \sum_{r=1}^R  \sum_{n=2}^{N^{(r)}} \sum_i^K \sum_j^K \overline{ z_{n-1, i}^{(r)} z_{nj}^{(r)} } \cdot \ln A_{ij} + {\rm const.}  \\
%
\intertext{ となり、 }
%
  q^*( {\bf A} )  &\propto  p( {\bf A} ) \times \prod_i^K \prod_j^K {A_{ij}}^{N_{ij}^R}  %\\
%
\intertext{ となる。ただし、$N_{ij}^R = \sum_{r=1}^R \sum_{n=2}^{N^{(r)}} \overline{ z_{n-1, i}^{(r)} z_{nj}^{(r)} }$。
ここで、事前分布として各 $i$ での $j$ に関する Dirichlet 分布 $p({\bf A}_i) \propto \prod_j^K {A_{ij}}^{(u_{ij}^A - 1)}$ を与えて規格化して、 }
%
  q^*({\bf A}_i)  &=  \frac{ \Gamma\left( u_{i0}^A + M_i^R \right) }{\displaystyle  \prod_j^K \Gamma\left( u_{ij}^A + N_{ij}^R \right) }\prod_j^K {A_{ij}}^{\left( u_{ij}^A + N_{ij}^R - 1 \right)}  
\end{align}
を得る。
ただし、$u_{ij}^A$ は Dirichlet 分布のハイパーパラメータ、$u_{i0}^A = \sum_j^K u_{ij}^A$, $M_i^R = \sum_j^K N_{ij}^R = \sum_{r=1}^R \sum_{n=2}^{N^{(r)}} \sum_j^K \overline{ z_{n-1, i}^{(r)} z_{nj}^{(r)} }$。

Dirichlet 分布の性質から、期待値
\begin{align}
  \overline{A_{ij}}  &=  \frac{ u_{ij}^A + N_{ij}^R }{ u_{i0}^A + M_i^R }  \\ %\label{eqn:avgA} \\
  \overline{\ln A_{ij}}  &=  \psi \left( u_{ij}^A + N_{ij}^R \right) - \psi\left( u_{i0}^A + M_i^R \right)  %\label{eqn:avgLnA}  
\end{align}
を得る。

\

$\phi$ に関しては、$p( x_m^{(r)} | {\bf z}_m^{(r)} )$ の関数形に依存するが、$r$ に関して独立な場合には、一般化できる。
\begin{align}
  \ln q^*( \phi )  &=  \ln p( \phi ) + \mathbb{E}_{\bf Z} \left[ \sum_{r=1}^R  \sum_{m=1}^{N^{(r)}} \ln p( x_m^{(r)} | {\bf z}_m^{(r)}, \phi ) \right] + {\rm const.}  \\
%
  &=  \ln p( \phi ) + \sum_{r=1}^R  \mathbb{E}_{\bf Z} \left[ \sum_{m=1}^{N^{(r)}} \ln p( x_m^{(r)} | {\bf z}_m^{(r)}, \phi ) \right] + {\rm const.}  
\end{align}
各 $r$ について、単独の HMM と同様に $\ln q^*( \phi )$ を求めて (事前分布は除いて) 和をとればよい (?)。


\subsection{ 変分下限 ${\cal L}(q)$ の算出 }

式 (\ref{eqn:vbHmmSimplidiedLq}) を既知のパラメータについて書き下すと、
\begin{align}
  {\cal L}(q)  =&  \mathbb{E}\bigl[\ln p({\boldsymbol \pi}) \bigr] + \mathbb{E}\bigl[\ln p({\bf A}) \bigr] + \mathbb{E}\bigl[\ln p(\phi) \bigr] + \mathbb{E}\bigl[\ln p( {\bf X}, {\bf Z} | \phi) \bigr]  \notag  \\
  &  - \mathbb{E}\bigl[\ln q({\boldsymbol \pi}) \bigr] - \mathbb{E}\bigl[\ln q({\bf A}) \bigr] - \mathbb{E}\bigl[\ln q( \phi ) \bigr] - \mathbb{E}\bigl[\ln q( {\bf Z} ) \bigr]  \\
%
  =&  \mathbb{E}\bigl[\ln p({\boldsymbol \pi}) \bigr] + \mathbb{E}\bigl[\ln p({\bf A}) \bigr] + \mathbb{E}\bigl[\ln p(\phi) \bigr] + \mathbb{E}\left[\sum_{r=1}^R \ln p( {\bf X}^{(r)}, {\bf Z}^{(r)} | \phi) \right]  \notag  \\
    &  - \mathbb{E}\bigl[\ln q({\boldsymbol \pi}) \bigr] - \mathbb{E}\bigl[\ln q({\bf A}) \bigr] - \mathbb{E}\bigl[\ln q( \phi ) \bigr] - \mathbb{E}\left[\sum_{r=1}^R \ln q( {\bf Z}^{(r)} ) \right]  \\
%
 =&  \mathbb{E}\bigl[\ln p({\boldsymbol \pi}) \bigr] + \mathbb{E}\bigl[\ln p({\bf A}) \bigr] + \mathbb{E}\bigl[\ln p(\phi) \bigr]  \notag  \\  
    &  - \mathbb{E}\bigl[\ln q({\boldsymbol \pi}) \bigr] - \mathbb{E}\bigl[\ln q({\bf A}) \bigr] - \mathbb{E}\bigl[\ln q( \phi ) \bigr] + \sum_{r=1}^R \sum_{n=1}^N \ln \tilde{c}_n^{(r)}  %\label{eqn:vbHmmLq}  
\end{align}
となるので、その他のパラメータに関して必要な項をそれぞれ計算する。

\paragraph{ $\mathbb{E}\bigl[\ln q({\boldsymbol \pi}) \bigr]$ : }
$q({\boldsymbol \pi}) = {\rm Dir}(\pi_i|u_i^\pi + \overline{z_{1i}^R})$ なので、Dirichlet  分布の性質から、
\begin{align}
  \mathbb{E}\bigl[\ln q({\boldsymbol \pi}) \bigr]  &=  \mathbb{E}_{\pi}\bigl[ \ln q({\boldsymbol \pi}) \bigr]  \\
%
  &=  \sum_i^K \Biggl[ (u_i^\pi + \overline{z_{1i}^R} - 1) \Biggl\{ \psi(u_i^\pi + \overline{z_{1i}^R}) - \psi\Biggl(u_0^\pi + \sum_i^K \overline{z_{1i}^R}\Biggr) \Biggr\} \Biggr] + \ln \Biggl\{ \frac{ \Gamma(u_0^\pi + \sum_i^K \overline{z_{1i}^R}) }{ \prod_i^K \Gamma(u_i^\pi + \overline{z_{1i}^R}) } \Biggr\}  \\
%
  &= \ln \Gamma(u_0^\pi + R) + \sum_i^K \Bigl[ (u_i^\pi + \overline{z_{1i}^R} - 1) \bigl\{ \psi(u_i^\pi + \overline{z_{1i}^R}) - \psi(u_0^\pi + R) \bigr\} - \ln \Gamma(u_i^\pi + \overline{z_{1i}^R}) \Bigr]  %\label{eqn:expLnQPi}
\end{align}

\paragraph{ $\mathbb{E}\bigl[\ln q({\bf A}) \bigr]$ : }
$q({\bf A}_i) = {\rm Dir}(A_{ij}|u_{ij}^A + N_{ij}^R)$ なので、Dirichlet  分布の性質から、
\begin{align}
  \mathbb{E}\bigl[\ln q({\bf A}) \bigr]  =\;&  \mathbb{E}_{A}\bigl[ \ln q({\bf A}) \bigr]  \\
%
  =\;&  \sum_i^K \mathbb{E}_{A_i}\bigl[ \ln q({\bf A}_i) \bigr]  \\
%
  =\;&  \sum_i^K \Biggl[ \sum_j^K \Biggl\{ (u_{ij}^A + N_{ij}^R - 1) \Bigl[ \psi(u_{ij}^A + N_{ij}^R) - \psi(u_{i0}^A + M_i^R) \Bigr] \Biggr\}  \notag  \\
  &  + \ln \Biggl\{ \frac{ \Gamma(u_{i0}^A + M_i^R) }{ \prod_j^K \Gamma\bigl(u_{ij}^A + N_{ij}^R\bigr) } \Biggr\} \Biggr]  \\
%
  =\;&  \sum_i^K \biggl[ \ln \Gamma(u_{i0}^A + M_i^R)  %
    + \sum_j^K \Bigl\{ \bigl(u_{ij}^A + N_{ij}^R - 1\bigr) \Bigl[ \psi\bigl(u_{ij}^A + N_{ij}^R\bigr) - \psi\bigl(u_{i0}^A + M_i^R\bigr) \Bigr]  \notag  \\
  &  - \ln \Gamma\bigl(u_{ij}^A + N_{ij}^R\bigr) \Bigr\} \biggr]  %\label{eqn:expLnQA}
\end{align}


%$\mathbb{E}\bigl[\ln p( \phi ) \bigr], \mathbb{E}\bigl[\ln q( \phi ) \bigr]$ については、関数形に依存するので、${\boldsymbol \pi}$, ${\bf A}$  と同様にそれぞれ計算する。


